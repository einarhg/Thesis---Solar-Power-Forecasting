\chapter{Introduction\label{cha:introduction}}

Solar power production is highly dependent on weather conditions \cite{lin_temporal_2020, lee_forecasting_2018, jaidee_very_2019, su_machine_2019, jang_solar_2016}. A single cloud can adversely impact the production of a solar farm. Such unreliability in a power production system is undesirable since production must match demand \cite{lee_forecasting_2018}. When solar power production falls, the power demand it was serving must be fulfilled by other means. Fossil fueled power stations are generally kept on standby, waiting to fulfil this demand \cite{lee_forecasting_2018}. These power stations often take hours to start, generally pollute more than other power stations and are expensive to run \cite{lee_forecasting_2018}.
Reliable and accurate forecasts of the fluctuations in solar power production can give power grid operators notice and insight into future power production \cite{lee_forecasting_2018}. This notice enables planning for the dips in power production and making arrangements for the future. This provides the tools to reduce the cost and pollution of power generation as well as increase the reliability of power delivery to consumers \cite{lee_forecasting_2018}.
Recent advancements in deep learning have unlocked new tools that have been shown to be effective in predicting the future behaviour of a system given enough relevant data.\\

%% \ifdraft only shows the text in the first argument if you are in draft mode.
%% These directions will disappear in other modes.
%\ifdraft{State the objectives of the exercise. Ask yourself:
%  \underline{Why} did I design/create the item? What did I aim to
%  achieve? What is the problem I am trying to solve?  How is my
%  solution interesting or novel?}{}


 
\section{Goals}
Previous methods utilized in the prediction of solar power production all have in common that they must effectively predict the weather so they can predict power output.

Here we explore using high-resolution weather forecast data from meteorological models, offloading a majority of this difficult task to the field of meteorology. Instead, we focus on correlating these forecasts with measurements and detecting and reducing temporal and spatial error in the general forecasts using newer methods like Temporal Fusion Transformers.
What is the quality of the predictions this methodology produces? Can we create a model which is useful to power producers in operating their systems more efficiently?


The thesis is structured as follows. First we talked about the motivations for the work. Next we discuss the background the work stands on. Next we cover the methods utilized, the data and tools used to achieve these goals. The proceeding chapter contains the results of the work, and finally we discuss the implications, strengths and flaws of the work, future work and end with conclusions.


%Here we explore whether better prediction accuracy can be reached
%if high quality, high resolution weather forecast data is used. This offloads a majority of the difficult task of predicting the weather from the responsibilities of the network. \\
%Can we construct a transformer based neural network which, by utilizing weather forecast data, is superior to neural networks which only utilize historical weather data?\\


%% Glossary is broken, do not use --foley
% \gls{auv}\footnote{Autonomous Undersea Vehicle}.

% Notice that there is now information on the AUV in the Index and Acronyms.
% It isn't in the \gls{glossary} because we didn't put it there.
%\index{AUV}
%}{}


